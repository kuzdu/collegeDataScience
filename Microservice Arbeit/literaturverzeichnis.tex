\section{Literaturverzeichnis}
% Literaturverweise werden durch den Befehl 'cite' und die
% Zeichenkette, die das zu zitierende Dokument markiert, gesetzt.
% Die Zeichenkette ist identisch zu der im Literaturdatenbankfile.
Studienanfänger der Physik benutzen oft \cite{gerthsen2013physik} als
Einführungstext\cite{xxxx} in die Experimentalphysik. Für \LaTeX{} verwenden wir
den Onlinetext \cite{xxxx}. Während oder nach der
Masterarbeit schreiben wir vielleicht auch mal eine Veröffentlichung
wie hier in \cite{heymans2006shear}.
%
% Setzen Sie den 'printbibliography' Befehl an die Stelle wo
% Sie das Literaturverzeichnis im Ausgabedokument sehen wollen.
\printbibliography
\end{document}
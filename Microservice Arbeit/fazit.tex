\section{Fazit}
Gelungen, weil grundlegende Microservices Architektur konnte umgesetzt werden. D.h. Datenbankenund Services gekoppelt, Gateway ud Authentifizierung. Das Ganze läuft im Livebetrieb mittels Docker. 

Mittels User Story Mapping und Bounded Contexts konnten Anforderungen sowie Grenzen der MS gut erfasst werden.

Schwierigkeiten in der Portvergabe.

\subsection{Ausblick}
- CI/CD 
- Loadbalancing
- ELK Stack
- Service Mash, Verknüpfung sehr aufwendig
- Kubernetes (Orchestrierung)
- Sichtbarkeit von Microservices (was soll alles erreichbar sein): Also nur das Gateway und/oder auch die Microservices
- Logging, Ausfallsicherheit, Informationen (Errors werden noch verschluckt)
- Health Check
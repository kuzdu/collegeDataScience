\section{Glossar}

Das Wörterverzeichnis dient als Nachschlagwerk für Begriffe, die bereits bekannt sein können oder im Laufe dieser Arbeit eingeführt werden. Zusätzlich gilt zu erwähnen, dass folgende Wörter als Synonm zueinander verwendet werden: Microservice/Service, Anfrage/Request/API-Call, Dockercontainer/Container. 

%TODO: Synonome Requests/Anfragen, Services/Microservices
\begin{longtable}{lp{10cm}}
	Begriff & Erklärung \\ \hline
	Deployen/Deployment & Ein Softwareprodukt updaten.  \\
	Hosten & Ein Softwareprodukt auf einem Server/Cloud bereitstellen. \\
	CI/CD & Continuous Integration/Continuous Delivery beschreibt den kontinuierlichen Prozess ein Softwareprodukt von der Entwicklung bis zur tatsächlichen Veröffentlichung zu begleiten.\\
	User Interface  & Oberfläche für Benutzer  \\
	VM & Virtuelle Maschine, Kapselung eines Rechnersystems innerhalb eines anderen  \\
	Legacy & Es handelt sich um Altsysteme, die eine Erneuerung benötigen oder abgelöst werden müssen. \\
	Technologiestack & Technologien die für ein Produkt oder in einer Firma verwendet werden  \\
	Downtime & Zeitpunkt, die eine Anwendung benötigt, bis sie neugestartet ist. Während der Downtime ist eine Anwendung nicht erreichbar.  \\
	Overhead/Technische Schuld & Technische Entscheidungen, die zu Mehrarbeit führen. \\
	Monolith & Eine Anwendung mit monolitsche Struktur. D.h. alles ist in einem. \\ 
	Pattern & Entwurfsmuster zum Bewältigen von Aufgaben.\\
	Domain Driven Design (DDD) & Pattern zum Abtrennen von Microservices .\\ 
	Bounded Contexts & Kontexte die zusammenhängen. \\
	Up-stream & Stellt dem down-stream Informationen bereit. \\
	Down-stream & Erhält vom up-stream Informationen., \\
	Client & Ein Gerät, welches Anfragen an die API stellt. \\
	API & Die Schnittstelle, mit der Clienten kommuniziert werden kann.\\
	REST & Ein Programmierparadigma für Webservices. \\
	Monitoring & Das Überwachen von Diensten oder Microservices.\\
	Logging & Das Aufzeichnen von sinnvollen Daten.\\
	Traffic & Datenverkehr, der zwischen der anfragenden und verarbeitenden Instanz entsteht.\\
	Timeout & Entsteht, wenn ein Dienst nicht erreichbar ist.\\
	Fallback & Eine Art Notfall-Verhalten, wenn das beabsichtige Verhalten nicht zutrifft.\\
	API-Route/Route & Pfad über den HTTP-Requests abgesetzt werden können.\\
	Message Broker/Eventbus & Vermittler, um Nachrichten zu versenden.  \\
	Subscribe & Nachrichten, welche beim Eventbus abonniert werden. \\
	Publish & Nachrichten, welche veröffentlicht werden. \\
	Load Balancer & Verteilt die Last/den Traffic auf verschiedene Services, um das Gesamtsystem zu entlasten.\\
	Backend & Anwendung, die für die Logik zuständig ist. Sie hat keine Oberfläche.\\
	Performance & Leistung des System, umso schneller die Performance ist, desto besser.\\
	Latenz & Verzögerungszeit, z.B. die Länge des Zeitintervalls vom Beginn bis zur tatsächlichen Reaktion.\\
	Middleware & Ein Knotenpunkt, den jeder Request durchlaufen muss. \\
	Parsen & Werte aus einem meist speziellen Format entnehmen und weiterverarbeiten. \\ 
	Enkodieren und Dekodieren & Daten in ein spezielles Format umwandeln oder zurückwandeln.\\
	\caption[Glossar]{Glossar}
\end{longtable}
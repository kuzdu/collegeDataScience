\section{Glossar}

Das Wörterverzeichnis dient als Nachschlagewerk für Begriffe, die bereits bekannt sein können oder im Laufe dieser Arbeit eingeführt werden. Zusätzlich gilt zu erwähnen, dass folgende Wörter als Synonym zueinander verwendet werden: Microservice/Service, Anfrage/Request/API Call, Dockercontainer/Container, Gatway/API Gateway. 

\begin{longtable}{|l|p{10cm}|}
	Begriff & Erklärung \\ \hline
	Deployen/Deployment & Ein Softwareprodukt updaten.  \\ \hline
	Hosten & Ein Softwareprodukt auf einem Server bereitstellen. \\\hline
	CI/CD & Continuous Integration/Continuous Delivery beschreibt den kontinuierlichen Prozess ein Softwareprodukt von der Entwicklung bis zur tatsächlichen Veröffentlichung zu begleiten.\\\hline
	User Interface (UI)  & Oberfläche für Benutzer.  \\\hline
	Virtuelle Maschine (VM) & Kapselung eines Rechnersystems innerhalb eines anderen.  \\ \hline
	Legacy (Erbe) & Es handelt sich um Altsysteme, die eine Erneuerung benötigen oder abgelöst werden müssen. \\ \hline
	Technologiestack & Technologien, die für ein Produkt oder in einer Firma verwendet werden.  \\ \hline
	Downtime (Stillzustand) & Zeitpunkt, den eine Anwendung benötigt, bis sie neu gestartet ist. Während der Downtime ist eine Anwendung nicht erreichbar.  \\ \hline
	Overhead/Technische Schuld & Technische Entscheidungen, die zu Mehrarbeit führen. \\ \hline
	Monolith & Eine Anwendung mit monolithischer Struktur. \\ \hline
	Pattern (Muster) & Entwurfsmuster zum Bewältigen von Aufgaben.\\ \hline
	Domain Driven Design (DDD) & Pattern zum Abtrennen von Microservices.\\ \hline
	Bounded Contexts & Kontexte, die aufgrund ihrer Aufgaben zusammenhängen. \\ \hline
	Up-stream & Stellt dem down-stream Informationen bereit. \\ \hline
	Down-stream & Erhält vom up-stream Informationen.\\ \hline
	Client & Ein Gerät, welches Anfragen an die API stellt. \\ \hline
	API & Die Schnittstelle einer Software, über die kommuniziert werden kann.\\ \hline
	REST & Ein Programmierparadigma für Webservices. \\ \hline
	Monitoring & Das Überwachen von Diensten oder Microservices.\\ \hline
	Logging & Das Aufzeichnen von Daten zum Nachvollziehen vergangener Aktionen.\\ \hline
	Traffic & Datenverkehr, der zwischen der anfragenden und verarbeitenden Instanz entsteht.\\ \hline
	Timeout & Entsteht, wenn ein Dienst nicht erreichbar ist.\\ \hline
	Fallback & Eine Art Notfallverhalten, wenn das beabsichtige Verhalten nicht eintrifft.\\ \hline
	API Route/Route & Pfad über den HTTP-Requests abgesetzt werden können.\\ \hline
	Message Broker/Eventbus & Vermittler, um Nachrichten zu versenden.  \\ \hline
	Subscribe (abonnieren) & Nachrichten, welche beim Eventbus abonniert werden. \\ \hline
	Publish & Nachrichten, welche veröffentlicht werden. \\ \hline
	Load Balancer & Verteilt die Last/den Traffic auf verschiedene Services, um das Gesamtsystem zu entlasten.\\ \hline
	Backend & Anwendung, die für die Logik zuständig ist. Sie hat keine Oberfläche.\\ \hline
	Performance & Leistung des Systems. Umso schneller die Performance ist, desto besser.\\ \hline
	Latenz & Verzögerungszeit, z.B. die Länge des Zeitintervalls vom Beginn bis zur tatsächlichen Reaktion.\\ \hline
	Middleware & Ein Knotenpunkt, den jeder Request durchlaufen muss. \\ \hline
	Parsen & Werte aus einem meist speziellen Format entnehmen und weiterverarbeiten. \\ \hline 
	Enkodieren und Dekodieren & Daten in ein spezielles Format umwandeln oder zurückwandeln.\\ \hline
	\caption[Glossar]{Glossar}
\end{longtable}
\pagebreak
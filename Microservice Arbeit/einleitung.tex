\section{Einleitung}\label{sec:einleitung}

Wenn ein Onlineshop genutzt wird, wirkt dieser wie eine einzelne Softwareanwendung. Es wird sich durch Artikel geklickt, der Warenkorb befüllt und schließlich eine Bestellung abgesendet. Wächst ein Onlineshop, wird dieser entsprechend komplexer und bietet zahlreiche Funktionalitäten. Auf Softwareebene entstehen dadurch Herausforderungen. Wie wird z.B. mit massiv steigenden Benutzerzahlen oder vermehrt hinterlegten Datensätzen umgegangen? Dieses Phänomen betrifft natürlich nicht nur Onlineshops, sondern alle Softwareanwendungen, die viel genutzt werden, wie z.B. Spiele- oder Streamingplattformen, soziale Netzwerke, Buchungsseiten für Hotels, Autos oder Flüge usw. Zusätzlich entstehen innerhalb des Unternehmens personelle Herausforderungen. An großer Software arbeiten in der Regel viele Mitarbeiter, diese müssen sich entsprechend organisieren und absprechen, damit sie nicht destruktiv arbeiten.\\

%Verweisen auf irgendwelche Quellen mit otto, google usw.
Abhilfe schafft die sogenannte Microservices-Architektur. Die Idee ist, eine große Softwareanwendung (Monolith) in viele kleine, eigenständig und in sich funktionierende Anwendungen (Microservices) zu zerteilen. Viele große Unternehmen wie Netflix, Amazon und Google bauen ihre Dienste bereits so auf. Aber auch deutsche Unternehmen wie Zalando, Otto oder Rewe nutzen diese Art der Architektur.\\

Für den außenstehenden Benutzer ist die Architektur irrelevant, d.h. die Anwendung hat dasselbe Erscheinungsbild und dieselbe Funktionalität. Auf technischer Ebene bietet die Microservice Architektur dagegen einfache Möglichkeiten der Skalierung, schnelle Updatezyklen und Agilität. Auch auf personeller Ebene kann die Organisation leichter werden, denn kleinere Teams sind verantwortlich für Microservices und können diesen entsprechend weiterentwickeln ohne andere Teams zu blockieren oder womöglich Fehler einzubauen. \\

%genauer noch drauf eingehen, Konzeption, implementierung, User Story Mappign
In der folgendenden Arbeit wird exemplarisch eine API nach Microservice Gesichtspunkten eruiert. Dafür wird zunächst untersucht, inwiefern sich der Monolith zu den Microservices unterscheidet und wie man Microservices untereinander abgrenzen kann. Ebenfalls wird die Schnittstellenkommunikation zwischen Microservices beleuchtet sowie die Herausforderung einer Authentifizeriung und Autorisierung. Zusätzlich wird die Infrastruktur zum Aufsetzen mittels Docker und die Erreichbarkeit aller Services durch ein API Gateway behandelt.\\ 

Anschließend werden Anforderungen, die an die Stirnraten API gestellt werden, mittels User Story Mapping erfasst und herausgearbeitet. Für die Umsetzung werden in der Konzeption verschiedene Technologien und Strategien gegeneinander abgewogen und darauf beruhend in der Implementierung verwendet.\\ 

Ziel ist es, eine lauffähige Stirnraten API basierend auf der Microservice-Architektur umzusetzen und der Stirnraten-App die Möglichkeit zur Kommunikation mit einem Backend zu geben. \\
\section{Konzept}
Das ist der Konzeptpart

\subsection{Anforderungen definieren}

\subsection{Macroarchitektonische Festlegungen}
- MySQL, Docker, selbe Begründung
- REST (Verlinkung auf Mobile Development-Modul)
- Bounded Context erstellen
- Aus sicht des User Story Mappings: 

\subsection{Wahl des API-Gateway}
- Ocelot 
- welches gibt es noch? 
- Eigenbau 
- Auth0
% Artikel lesen und Grundlagen rausarbeiten: https://docs.microsoft.com/de-de/dotnet/standard/microservices-architecture/secure-net-microservices-web-applications/

% Guter Artikel bzgl. OAuth2: https://medium.com/google-cloud/understanding-oauth2-and-building-a-basic-authorization-server-of-your-own-a-beginners-guide-cf7451a16f66

\subsection{Wahl der Kommunikation}
- Erwähnen, dass asynchrone Kommunikation verwendet wird, aufgrund der Vorteile aus den Grundlagen 
- Technologie: Masstransit (RabbitMQ) vs. Kafka

\subsection{Wahl der Authenfizierung/Authorisierung}
- OAuth2: Authentication Framework (mit OpenId Connect?) -> Spricht die Zeit, Komplexität und so eine Komplexität gar nicht benötigt 
- JWT: Also Protocol
- Dritteranbieter Dienst -> Auth0 (zu teuer)

